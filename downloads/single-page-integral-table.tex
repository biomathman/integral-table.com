%&LaTeX
\documentclass[11pt,twoside]{article}
\usepackage[left=0.25in,right=0.25in,top=0.25in,bottom=0.25in]{geometry} 
\geometry{paperwidth=10.625in,paperheight=13.75in}  
\setlength{\paperheight}{13.75in}
\setlength{\paperwidth}{10.625in}
              
\usepackage{graphicx}
\usepackage{amssymb}
\usepackage{epstopdf}
\usepackage{amsmath}
\usepackage{textcomp}
\usepackage{fancyhdr}
\usepackage{hyperref}
\hypersetup{
    pdftitle={Integral Table from http://integral-table.com},    % title
    pdfauthor={Shapiro},     % author
    pdfsubject={Table of Integrals},   % subject of the document
    pdfcreator={pdftex},   % creator of the document
    pdfproducer={Texmaker}, % producer of the document
    pdfkeywords={CSUN, Integrals, Table of Integrals, Math 280, Math 351, Differential Equations}, % list of keywords
    colorlinks=true,       % false: boxed links; true: colored links
    linkcolor=red,          % color of internal links
    citecolor=red,        % color of links to bibliography
    filecolor=red,      % color of file links
    urlcolor=red           % color of external links
}

\usepackage{multicol}

\date{} % delete this line to display the current date                                                                                             

\renewcommand{\thefootnote}{\fnsymbol{footnote}}
\usepackage[hang,flushmargin]{footmisc}


\begin{document}


\begin{center}
\LARGE
Table of Integrals\footnote{\textcopyleft \ 2014. From  \url{http://integral-table.com}, last revised \today. This material 
is provided as is without warranty or representation about the accuracy, correctness or suitability of the material for any purpose, and is licensed under the Creative Commons Attribution-Noncommercial-ShareAlike 3.0 United States License. To view a copy of this license, visit \url{http://creativecommons.org/licenses/by-nc-sa/3.0/} or send a letter to Creative Commons, 171 Second Street, Suite 300, San Francisco, California, 94105, USA. } 



\end{center}
\normalsize
\begin{multicols}{3}

\begin{footnotesize}
\begin{center} \textbf{Basic Forms}
\end{center}

\begin{equation}
\int x^n dx = \frac{1}{n+1}x^{n+1}
\end{equation}

\begin{equation}
\int \frac{1}{x}dx = \ln |x|
\end{equation}

\begin{equation}
\int u dv = uv - \int v du
\end{equation}

\begin{equation}
\int \frac{1}{ax+b}dx = \frac{1}{a} \ln |ax + b| 
\end{equation}

\begin{center} \textbf {Integrals of Rational Functions} \end{center}

\begin{equation}
\int \frac{1}{(x+a)^2}dx = -\frac{1}{x+a}
\end{equation}

\begin{equation}
\int (x+a)^n dx = \frac{(x+a)^{n+1}}{n+1}, n\ne -1
\end{equation}

\begin{equation}
\int x(x+a)^n dx = \frac{(x+a)^{n+1} ( (n+1)x-a)}{(n+1)(n+2)}
\end{equation}

\begin{equation}
\int \frac{1}{1+x^2}dx = \tan^{-1}x
\end{equation}

\begin{equation}
\int \frac{1}{a^2+x^2}dx = \frac{1}{a}\tan^{-1}\frac{x}{a}
\end{equation}

\begin{equation}
\int \frac{x}{a^2+x^2}dx = \frac{1}{2}\ln|a^2+x^2|
\end{equation}

\begin{equation}
\int \frac{x^2}{a^2+x^2}dx = x-a\tan^{-1}\frac{x}{a}
\end{equation}

\begin{equation}
\int \frac{x^3}{a^2+x^2}dx = \frac{1}{2}x^2-\frac{1}{2}a^2\ln|a^2+x^2|
\end{equation}

\begin{equation}
\int \frac{1}{ax^2+bx+c}dx = \frac{2}{\sqrt{4ac-b^2}}\tan^{-1}\frac{2ax+b}{\sqrt{4ac-b^2}}
\end{equation}

\begin{equation}
\int \frac{1}{(x+a)(x+b)}dx = \frac{1}{b-a}\ln\frac{a+x}{b+x}, \text{ } a\ne b
\end{equation}

\begin{equation}
\int \frac{x}{(x+a)^2}dx = \frac{a}{a+x}+\ln |a+x|
\end{equation}


\begin{align}
\int \frac{x}{ax^2+bx+c}dx &= \frac{1}{2a}\ln|ax^2+bx+c| \nonumber
\\&-\frac{b}{a\sqrt{4ac-b^2}}\tan^{-1}\frac{2ax+b}{\sqrt{4ac-b^2}}
\end{align}
\begin{center} \textbf{Integrals with Roots} \end{center}

\begin{equation}
\int \sqrt{x-a} dx = \frac{2}{3}(x-a)^{3/2}
\end{equation}

\begin{equation}
\int \frac{1}{\sqrt{x\pm a}} dx = 2\sqrt{x\pm a} 
\end{equation}

\begin{equation}
\int \frac{1}{\sqrt{a-x}} dx = -2\sqrt{a-x} 
\end{equation}

\begin{equation}
\int x\sqrt{x-a} dx = \frac{2}{3}a(x-a)^{3/2}+\frac{2}{5}(x-a)^{5/2}
\end{equation}

\begin{equation}
\int \sqrt{ax+b}dx = \left(\frac{2b}{3a}+\frac{2x}{3}\right)\sqrt{ax+b}
\end{equation}

\begin{equation}
\int (ax+b)^{3/2}dx =\frac{2}{5a}(ax+b)^{5/2}
\end{equation}

\begin{equation}
\int \frac{x}{\sqrt{x\pm a} } dx = \frac{2}{3}(x\mp 2a)\sqrt{x\pm a}
\end{equation}

\begin{align}
\int \sqrt{\frac{x}{a-x}}dx &=  -\sqrt{x(a-x)}
%\nonumber \\&
-a\tan^{-1}\frac{\sqrt{x(a-x)}}{x-a}
\end{align}

\begin{align}
\int \sqrt{\frac{x}{a+x}}dx &=  \sqrt{x(a+x)} 
%\nonumber \\& 
-a\ln \left [ \sqrt{x} + \sqrt{x+a}\right] 
\end{align}

\begin{align}
\int &x \sqrt{ax + b}dx =
%\nonumber \\& 
\frac{2}{15 a^2}(-2b^2+abx + 3 a^2 x^2)
\sqrt{ax+b}
\end{align}

\begin{align}
\int \sqrt{x(ax+b)} dx &= \frac{1}{4a^{3/2}}\left[(2ax + b)\sqrt{ax(ax+b)} \right. \nonumber
\\& \left.
-b^2 \ln \left| a\sqrt{x} + \sqrt{a(ax+b)} \right| \right ] 
\end{align}

\begin{align}
\int \sqrt{x^3(ax+b)} dx &=\left [ 
\frac{b}{12a}-
\frac{b^2}{8a^2x}+
\frac{x}{3}\right] 
\sqrt{x^3(ax+b)} \nonumber \\& + 
\frac{b^3}{8a^{5/2}}\ln \left | a\sqrt{x} + \sqrt{a(ax+b)} \right |
\end{align}

\begin{align}
\int\sqrt{x^2 \pm a^2} dx &= \frac{1}{2}x\sqrt{x^2\pm a^2} 
%\nonumber \\&
\pm\frac{1}{2}a^2 \ln \left | x + \sqrt{x^2\pm a^2} \right | 
\end{align}

\begin{align}
\int  \sqrt{a^2 - x^2} dx &= \frac{1}{2} x \sqrt{a^2-x^2} 
%\nonumber \\ &
+\frac{1}{2}a^2\tan^{-1}\frac{x}{\sqrt{a^2-x^2}}
\end{align}

\begin{equation}
\int  x \sqrt{x^2 \pm a^2} dx= \frac{1}{3}\left ( x^2 \pm a^2 \right)^{3/2} 
\end{equation}

\begin{equation}
\int \frac{1}{\sqrt{x^2 \pm a^2}} dx = \ln \left | x + \sqrt{x^2 \pm a^2} \right | 
\end{equation}

\begin{equation}
\int \frac{1}{\sqrt{a^2 - x^2}} dx = \sin^{-1}\frac{x}{a} 
\end{equation}

\begin{equation}
\int \frac{x}{\sqrt{x^2\pm a^2}}dx = \sqrt{x^2 \pm a^2} 
\end{equation}

\begin{equation}
\int \frac{x}{\sqrt{a^2-x^2}}dx = -\sqrt{a^2-x^2} 
\end{equation}

\begin{align}
\int \frac{x^2}{\sqrt{x^2 \pm a^2}} dx &= \frac{1}{2}x\sqrt{x^2 \pm a^2}
%\nonumber \\ &
\mp \frac{1}{2}a^2 \ln \left| x + \sqrt{x^2\pm a^2} \right | 
\end{align}

\begin{align}
\int &\sqrt{a x^2 + b x + c} dx = 
\frac{b+2ax}{4a}\sqrt{ax^2+bx+c}
\nonumber \\ &
+
\frac{4ac-b^2}{8a^{3/2}}\ln \left| 2ax + b + 2\sqrt{a(ax^2+bx^+c)}\right |
\end{align}

\begin{align}
\int &x \sqrt{a x^2 + bx + c} = \frac{1}{48a^{5/2}}\left ( 
2 \sqrt{a} \sqrt{ax^2+bx+c}
\right . \nonumber \\ & 
 \times \left( -3b^2 + 2 abx + 8 a(c+ax^2) \right)
 \nonumber \\ & \left.
 + 3(b^3-4abc)\ln \left|b + 2ax + 2\sqrt{a}\sqrt{ax^2+bx+c} \right| \right)
\end{align}

\begin{align}
\int&\frac{1}{\sqrt{ax^2+bx+c}}dx=
%==\nonumber \\&
\frac{1}{\sqrt{a}}\ln \left| 2ax+b + 2 \sqrt{a(ax^2+bx+c)} \right | 
\end{align}

\begin{align}
\int &\frac{x}{\sqrt{ax^2+bx+c}}dx=
\frac{1}{a}\sqrt{ax^2+bx + c} \nonumber \\&
-
\frac{b}{2a^{3/2}}\ln \left| 2ax+b + 2 \sqrt{a(ax^2+bx+c)} \right |
\end{align}

\begin{equation}
\int\frac{dx}{(a^2+x^2)^{3/2}}=\frac{x}{a^2\sqrt{a^2+x^2}}
\end{equation}

\columnbreak

\begin{center} \textbf{Integrals with Logarithms} \end{center}

\begin{equation}
\int \ln ax dx = x \ln ax - x 
\end{equation}

\begin{equation}
\int \frac{\ln ax}{x} dx = \frac{1}{2}\left ( \ln ax \right)^2 
\end{equation}

\begin{equation}
\int \ln (ax + b) dx = \left ( x + \frac{b}{a} \right) \ln (ax+b) - x , a\ne 0
\end{equation}

\begin{align}
\int \ln  ( x^2 + a^2 )\hspace{.5ex}\text{dx} = x \ln (x^2 + a^2  ) +2a\tan^{-1} \frac{x}{a} - 2x 
\end{align}

\begin{align}
\int \ln  ( x^2 - a^2 )\hspace{.5ex}\text{dx} = x \ln (x^2 - a^2  ) +a\ln \frac{x+a}{x-a} - 2x \end{align}

\begin{align}
\int \ln & \left ( ax^2 + bx + c\right) dx  = \frac{1}{a}\sqrt{4ac-b^2}\tan^{-1}\frac{2ax+b}{\sqrt{4ac-b^2}}
\nonumber \\ & -2x
 + \left( \frac{b}{2a}+x \right )\ln \left (ax^2+bx+c \right) 
\end{align}

\begin{align}
\int x \ln (ax + b) dx &= \frac{bx}{2a}-\frac{1}{4}x^2 \nonumber
\\&
+\frac{1}{2}\left(x^2-\frac{b^2}{a^2}\right)\ln (ax+b) 
\end{align}

\begin{align}
\int x \ln \left ( a^2 - b^2 x^2 \right ) dx &= -\frac{1}{2}x^2+ \nonumber
\\&
\frac{1}{2}\left( x^2 - \frac{a^2}{b^2} \right ) \ln \left (a^2 -b^2 x^2 \right) 
\end{align}

 
\begin{center} \textbf{Integrals with Exponentials} \end{center}

\begin{equation}
\int e^{ax} dx = \frac{1}{a}e^{ax} 
\end{equation}

\begin{align}
\int \sqrt{x} e^{ax} dx &= \frac{1}{a}\sqrt{x}e^{ax} 
+\frac{i\sqrt{\pi}}{2a^{3/2}}
\text{erf}\left(i\sqrt{ax}\right), \nonumber \\&
\text{ where erf}(x)=\frac{2}{\sqrt{\pi}}\int_0^x e^{-t^2}dt
\end{align}

\begin{equation}
\int x e^x dx = (x-1) e^x 
\end{equation}

\begin{equation}
\int x e^{ax} dx = \left(\frac{x}{a}-\frac{1}{a^2}\right) e^{ax} 
\end{equation}

\begin{equation}
\int x^2 e^{x} dx = \left(x^2 - 2x + 2\right) e^{x}
\end{equation}

\begin{equation}
\int x^2 e^{ax} dx = \left(\frac{x^2}{a}-\frac{2x}{a^2}+\frac{2}{a^3}\right) e^{ax} 
\end{equation}

\begin{equation}
\int x^3 e^{x} dx = \left(x^3-3x^2 + 6x - 6\right) e^{x} 
\end{equation}
 
\begin{equation}
\int x^n e^{ax}\hspace{1pt}\text{d}x = \dfrac{x^n e^{ax}}{a} - 
\dfrac{n}{a}\int x^{n-1}e^{ax}\hspace{1pt}\text{d}x
\end{equation} 
 
\begin{equation}
\begin{split}
\int x^n e^{ax}\hspace{2pt}\text{d}x = \frac{(-1)^n}{a^{n+1}}\Gamma[1+n,-ax], \\
 \text{ where } \Gamma(a,x)=\int_x^{\infty} t^{a-1}e^{-t}\hspace{2pt}\text{d}t
 \end{split}
 \end{equation}

\begin{equation}
\int e^{ax^2}\hspace{1pt}\text{d}x = -\frac{i\sqrt{\pi}}{2\sqrt{a}}\text{erf}\left(ix\sqrt{a}\right) 
\end{equation}

\begin{equation}
\int e^{-ax^2}\hspace{1pt}\text{d}x = \frac{\sqrt{\pi}}{2\sqrt{a}}\text{erf}\left(x\sqrt{a}\right) 
\end{equation}

\begin{equation}
\int x e^{-ax^2}\ \text{dx} = -\dfrac{1}{2a}e^{-ax^2} 
\end{equation}

\begin{equation}
\int x^2 e^{-ax^2}\ \text{dx} = \dfrac{1}{4}\sqrt{\dfrac{\pi}{a^3}}\text{erf}(x\sqrt{a}) -\dfrac{x}{2a}e^{-ax^2}
\end{equation}

\columnbreak
\begin{center} \textbf {Integrals with Trigonometric Functions} \end{center}

\begin{equation}
\int \sin ax dx = -\frac{1}{a} \cos ax 
\end{equation}

\begin{equation}
\int \sin^2 ax dx = \frac{x}{2} - \frac{\sin 2ax} {4a} 
\end{equation}

\begin{align}
\int &\sin^n ax dx =
\nonumber \\ &
 -\frac{1}{a}{\cos ax} \hspace{2mm}{_2F_1}\left[
\frac{1}{2}, \frac{1-n}{2}, \frac{3}{2}, \cos^2 ax
\right] 
\end{align}

\begin{equation}
\int \sin^3 ax dx = -\frac{3 \cos ax}{4a} + \frac{\cos 3ax} {12a} 
\end{equation}

\begin{equation}
\int \cos ax dx= \frac{1}{a} \sin ax 
\end{equation}

\begin{equation}
\int \cos^2 ax dx = \frac{x}{2}+\frac{ \sin 2ax}{4a} 
\end{equation}

\begin{align}
\int \cos^p ax dx & = -\frac{1}{a(1+p)}{\cos^{1+p} ax} \times 
\nonumber \\ &
{_2F_1}\left[
\frac{1+p}{2}, \frac{1}{2}, \frac{3+p}{2}, \cos^2 ax
\right] 
\end{align}

\begin{equation}
\int \cos^3 ax dx = \frac{3 \sin ax}{4a}+\frac{ \sin 3ax}{12a} 
\end{equation}

\begin{align}
\int \cos ax \sin bx dx &= \frac{\cos[(a-b) x]}{2(a-b)} -
%\nonumber \\ &
 \frac{\cos[(a+b)x]}{2(a+b)} , a\ne b
\end{align}

\begin{align}
\int \sin^2 ax \cos bx dx &= 
-\frac{\sin[(2a-b)x]}{4(2a-b)} \nonumber \\ & 
+ \frac{\sin bx}{2b} 
- \frac{\sin[(2a+b)x]}{4(2a+b)}
\end{align}

\begin{equation}
\int \sin^2 x \cos x dx = \frac{1}{3} \sin^3 x
\end{equation}

\begin{align}
\int \cos^2 ax \sin bx dx &= \frac{\cos[(2a-b)x]}{4(2a-b)} 
- \frac{\cos bx}{2b}
\nonumber \\ &
 - \frac{\cos[(2a+b)x]}{4(2a+b)}
\end{align}

\begin{equation}
\int \cos^2 ax \sin ax dx = -\frac{1}{3a}\cos^3{ax} 
\end{equation}

\begin{align}
\int \sin^2 ax \cos^2 bx dx &= \frac{x}{4}
-\frac{\sin 2ax}{8a}-
\frac{\sin[2(a-b)x]}{16(a-b)}
\nonumber \\ &
+\frac{\sin 2bx}{8b}-
\frac{\sin[2(a+b)x]}{16(a+b)}
\end{align}

\begin{equation}
\int \sin^2 ax \cos^2 ax dx = \frac{x}{8}-\frac{\sin 4ax}{32a}
\end{equation}

\begin{equation}
\int \tan ax dx = -\frac{1}{a} \ln \cos ax 
\end{equation}

\begin{equation}
\int \tan^2 ax dx = -x + \frac{1}{a} \tan ax 
\end{equation}

\begin{align}
\int &\tan^n ax dx = 
\frac{\tan^{n+1} ax }{a(1+n)} \times \nonumber \\ &
 {_2}F_1\left( \frac{n+1}{2}, 
1, \frac{n+3}{2}, -\tan^2 ax \right) 
\end{align}

\begin{equation}
\int \tan^3 ax dx = \frac{1}{a} \ln \cos ax + \frac{1}{2a}\sec^2 ax 
\end{equation}

\begin{align}
\int \sec x dx &= \ln | \sec x + \tan x | = 2 \tanh^{-1} \left (\tan \frac{x}{2} \right) 
\end{align}

\begin{equation}
\int \sec^2 ax dx = \frac{1}{a} \tan ax 
\end{equation}

\begin{equation}
\int \sec^3 x \hspace{2pt}\text{dx} = \frac{1}{2} \sec x \tan x + \frac{1}{2}\ln | \sec x + \tan x |
\end{equation}

\begin{equation}
\int \sec x \tan x dx = \sec x 
\end{equation}

\begin{equation}
\int \sec^2 x \tan x dx = \frac{1}{2} \sec^2 x 
\end{equation}

\begin{equation}
\int \sec^n x \tan x dx = \frac{1}{n} \sec^n x , n\ne 0
\end{equation}

\begin{equation}
\int \csc x dx = \ln \left | \tan \frac{x}{2} \right|  = \ln | \csc x - \cot x| + C
\end{equation}

\begin{equation}
\int \csc^2 ax dx = -\frac{1}{a} \cot ax 
\end{equation}

\begin{equation}
\int \csc^3 x dx = -\frac{1}{2}\cot x \csc x + \frac{1}{2} \ln | \csc x - \cot x | 
\end{equation}

\begin{equation}
\int \csc^nx \cot x dx = -\frac{1}{n}\csc^n x, n\ne 0
\end{equation}

\begin{equation}
\int \sec x \csc x dx = \ln | \tan x | 
\end{equation}


\begin{center} \textbf{Products of Trigonometric Functions and Monomials} \end{center}

\begin{equation}
\int x \cos x dx = \cos x + x \sin x 
\end{equation}

\begin{equation}
\int x \cos ax dx = \frac{1}{a^2} \cos ax + \frac{x}{a} \sin ax 
\end{equation}

\begin{equation}
\int x^2 \cos x dx = 2 x \cos x + \left ( x^2 - 2 \right ) \sin x 
\end{equation}

\begin{equation}
\int x^2 \cos ax dx = \frac{2 x \cos ax }{a^2} + \frac{ a^2 x^2 - 2  }{a^3} \sin ax 
\end{equation}

\begin{align}
\int  x^n cos x dx &= 
-\frac{1}{2}(i)^{n+1}\left [ \Gamma(n+1, -ix) 
\right . \nonumber \\ & \left .
+ (-1)^n \Gamma(n+1, ix)\right] 
\end{align}

\begin{align}
\int x^n cos ax dx &=
 \frac{1}{2}(ia)^{1-n}\left [ (-1)^n  \Gamma(n+1, -iax) 
 \right. \nonumber \\ & \left.
 -\Gamma(n+1, ixa)\right] 
\end{align}

\begin{equation}
\int x \sin x dx = -x \cos x + \sin x 
\end{equation}

\begin{equation}
\int x \sin ax dx = -\frac{x \cos ax}{a} + \frac{\sin ax}{a^2} 
\end{equation}

\begin{equation}
\int x^2 \sin x dx = \left(2-x^2\right) \cos x + 2 x \sin x
\end{equation}

\begin{equation}
\int x^2 \sin ax dx =\frac{2-a^2x^2}{a^3}\cos ax +\frac{ 2 x \sin ax}{a^2} 
\end{equation}

\begin{align}
\int x^n \sin x dx &= -\frac{1}{2}(i)^n\left[ \Gamma(n+1, -ix) 
%\right. \nonumber \\ & \left.
 - (-1)^n\Gamma(n+1, -ix)\right] 
\end{align}

 
\begin{center} \textbf{Products of Trigonometric Functions and Exponentials} \end{center}

\begin{equation}
\int e^x \sin x dx = \frac{1}{2}e^x (\sin x - \cos x) 
\end{equation}

\begin{equation}
\int e^{bx} \sin ax dx = \frac{1}{a^2+b^2}e^{bx} (b\sin ax - a\cos ax) 
\end{equation}

\begin{equation}
\int e^x \cos x dx = \frac{1}{2}e^x (\sin x + \cos x)  
\end{equation}

\begin{equation}
\int e^{bx} \cos ax dx = \frac{1}{a^2 + b^2} e^{bx} ( a \sin ax + b \cos ax ) 
\end{equation}

\begin{equation}
\int x e^x \sin x dx = \frac{1}{2}e^x (\cos x - x \cos x + x \sin x) 
\end{equation}

\begin{equation}
\int x e^x \cos x dx = \frac{1}{2}e^x (x \cos x 
- \sin x + x \sin x) 
\end{equation}

\begin{center} \textbf{Integrals of Hyperbolic Functions} \end{center}

\begin{equation}
\int \cosh ax dx =\frac{1}{a} \sinh ax 
\end{equation}

\begin{align}
\int e^{ax} & \cosh bx dx = \nonumber \\ &
\begin{cases}
\displaystyle{\frac{e^{ax}}{a^2-b^2} }[ a \cosh bx - b \sinh bx ]  & a\ne b \\
\displaystyle{\frac{e^{2ax}}{4a} + \frac{x}{2}}  & a = b
\end{cases}
\end{align}

\begin{equation}
\int \sinh ax dx = \frac{1}{a} \cosh ax 
\end{equation}

\begin{align}
\int e^{ax}& \sinh bx dx = \nonumber \\ &
\begin{cases}
\displaystyle{\frac{e^{ax}}{a^2-b^2} }[ -b \cosh bx + a \sinh bx ]  & a\ne b \\
\displaystyle{\frac{e^{2ax}}{4a} - \frac{x}{2}}  & a = b
\end{cases}
\end{align}

\begin{align}
\int & e^{ax} \tanh bx dx = \nonumber \\ &
\begin{cases}
\displaystyle{ \frac{ e^{(a+2b)x}}{(a+2b)} 
{_2F_1}\left[ 1+\frac{a}{2b},1,2+\frac{a}{2b}, -e^{2bx}\right] }& \\
\displaystyle{
\hspace{1cm}-\frac{1}{a}e^{ax}{_2F_1}\left[ \frac{a}{2b},1,1E, -e^{2bx}\right]
}
 & a\ne b \\
\displaystyle{\frac{e^{ax}-2\tan^{-1}[e^{ax}]}{a} } & a = b
\end{cases}
\end{align}

\begin{equation}
\int  \tanh ax\hspace{1.5pt} dx =\frac{1}{a} \ln \cosh ax 
\end{equation}

\begin{align}
\int \cos ax \cosh bx dx &= 
\frac{1}{a^2 + b^2} \left[
a \sin ax \cosh bx  \right . \nonumber \\ & \left. + b \cos ax \sinh bx
\right] 
\end{align}

\begin{align}
\int \cos ax \sinh bx dx& = 
\frac{1}{a^2 + b^2} \left[
b \cos ax \cosh bx +
\right . \nonumber \\ & \left .
 a \sin ax \sinh bx
\right] 
\end{align}

\begin{align}
\int \sin ax \cosh bx dx &= 
\frac{1}{a^2 + b^2} \left[
-a \cos ax \cosh bx +
\right . \nonumber \\ & \left .
 b \sin ax \sinh bx
\right] 
\end{align}

\begin{align}
\int \sin ax \sinh bx dx &= 
\frac{1}{a^2 + b^2} \left[
b \cosh bx \sin ax -
\right . \nonumber \\ & \left .
 a \cos ax \sinh bx
\right] 
\end{align}

\begin{equation}
\int \sinh ax \cosh ax dx= 
\frac{1}{4a}\left[ 
-2ax + \sinh 2ax \right]
\end{equation}

\begin{align}
\int \sinh ax \cosh bx dx&= 
\frac{1}{b^2-a^2}\left[ 
b \cosh bx \sinh ax 
\right . \nonumber \\ & \left .
- a \cosh ax \sinh bx \right]
\end{align}


\end{footnotesize}
\end{multicols}
\end{document}  